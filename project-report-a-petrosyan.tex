% Created 2019-12-03 Tue 20:08
\documentclass[11pt, reprint]{revtex4-2}
%\usepackage[autocite=superscript,style=numeric]{biblatex}
%\addbibresource{bibliography.bib}
\usepackage[utf8]{inputenc}
\usepackage[T1]{fontenc}
\usepackage{fixltx2e}
\usepackage{graphicx}
\usepackage{longtable}
\usepackage{float}
\usepackage{wrapfig}
\usepackage{rotating}
\usepackage[normalem]{ulem}
\usepackage{amsmath}
\usepackage{textcomp}
\usepackage{marvosym}
%\usepackage{multicol}
\usepackage{wasysym}
\usepackage{amssymb}
\usepackage{hyperref}

\usepackage[ruled,vlined]{algorithm2e}
\usepackage[a4paper, total={7.5in, 10in}]{geometry}
\tolerance=1000


\hypersetup{
  pdfkeywords={},
  pdfsubject={},
  pdfcreator={Emacs 25.2.2 (Org mode 8.2.10)}}
\begin{document}

\title{Cosmological parameter estimation using Bayesian accelerated machine learning.}
\author{Aleksandr Petrosyan}
\noaffiliation
\date{\today}
\renewcommand*{\bibfont}{\small}


\begin{abstract}In the era of precision cosmology, physicists use
  astronomical data to extract precise measurements of our universe's
  global properties. In order to do this, we develop of advanced
  inference and machine learning algorithms.  Cosmological parameter
  estimation is generally performed using Markov-Chain Monte-Carlo
  algorithms, such as Metropolis-Hastings, Hamiltonian Monte-Carlo or
  Nested sampling. This project will focus on extending recent work by
  the astrophysics group on prior re-partitioning to allow nested
  sampling to take the additional information usually provided to
  other algorithms into account, for accelerating inference. Such an
  extension for nested sampling would be highly desirable, and is the
  basis for the following research project.
\end{abstract}

\maketitle

\vspace{0.9in}
%\begin{multicols}{2}
\section{Introduction}\label{sec:introduction}

The standard cosmological model for the expansion and origin of the
universe is the accepted $\Lambda$CDM\cite{Condon2018} (Cold Dark
Matter). It has six major parameters: \emph{physical baryon density
  parameter; physical dark matter density parameter; the age of the
  universe; scalar spectral index; curvature fluctuation amplitude;
  and reionization optical depth}, which we will try to estimate. The
data from which we estimate the parameters \cite[See. ]{planck},
often contain a plethora of other less-important data e.g.~the
calibration of the equipment known, as nuisance parameters, in which
case the parameter space can have more than 40 dimensions. The main
issues with said nuisance parameters is that they cannot easily be
decoupled from the cosmological parameters of $\Lambda$CDM.

Among the possible approaches to parameter estimation, of particular
interest is Bayesian inference, which allows us not only to estimate
the likelihood of the parameters having particular values, but also
determine the validity of our physical model. As an added bonus,
Bayesian inference automatically accounts for nuisance parameters in
its formalism, including them in the analysis, but excluding them from
the end results.Bayesian inference in such a high dimensional
parameter space is carried out using a form of Markov-Chain
Monte-Carlo integration, the most common example being the
Metropolis-Hastings algorithm\cite{Metropolis}, or more recently
Nested Sampling\cite{polychord,skilling2006}.

In the following sections we shall outline the basics of Bayesian
parameter fitting (Section.~\ref{sec-2}), the basics of Nested
Sampling (Section.~\ref{sec-4}) and outline the methods of Prior
re-partitioning applied to Nested sampling (Section.~\ref{sec-4-2}).
\goodbreak%
\section{Bayesian parameter estimation}\label{sec-2}

Let's assume that a scientific theory has a model for a process \(m \)
which has \( n \) parameters \( \lbrace \theta \rbrace \) (we shall drop the braces
from now on). The real world observations give us a some data \( D
\). To verify the theory we're interested in the posterior:
\( {\cal P} = P(\theta | D, M) \) and the evidence
\( Z = P ( D | M ) \). Usually, the predictions of the model can
straightforwardly give us two things: the prior
\( \pi (\theta) = P (\theta | D)\) and the Likelihood
\( L(\theta) = P ( D | \theta, M) \). These quantities are linked via Bayes'
theorem: \[ Z {\cal P}(\theta) = L (\theta) \pi(\theta), \] which is a straightforward
result, with profound implications
\cite[p.~31]{jeffreys2010scientific}.

It is important to note the significance of \( Z
\). Some\cite[p.~32]{jeffreys2010scientific} regard it as a simple
normalisation factor, and deem it completely irrelevant to our
analysis, indeed even the name ``Evidence'' is, while agreed-upon, not
standardised like the terms ``prior'' and ``posterior''. Of course
it's an important factor determining the fitness of the $\Lambda$CDM model,
and we need it just as much if not more than the posterior
distribution.

To evaluate the \emph{evidence}, we shall make use of its definition
as a normalisation factor:

\begin{equation}
Z  = \int L (\theta) \pi (\theta) d \theta.
\end{equation}

For a completely new theory, determining the probability of each
parameter given the model is difficult, so the prior is usually
uniform within some
constraints\cite{skilling2006, polychord}. The probability of
data, conditional on the data and the model is usually a Gaussian
distribution, due to the central limit theorem.  Thus we can both
estimate how well our model fits the data, i.e.~how certain are we
that the universe is indeed $\Lambda$CDM and determine the distribution of
the parameters.

The above task is more computationally expensive thus the scientific
community rarely if ever goes beyond simple approximate but
computationally inexpensive parameter fitting algorithms. For example,
most experimental results are quoted with a single symmetric
uncertainty value (e.g.~\(a = 1.0 \pm 0.2 \)), and model suitability,
if investigated at all, is done via the \( \chi^2 \)
test\cite{Pearson1900}, which makes further assumptions about the
nature of experimental data distribution.

In Section~\ref{sec-3}, we shall explore various approaches to
performing full Bayesian inference.

\section{Evaluating the evidence}\label{sec-3}

\subsection{Full rasterisation}\label{sec-3-1}

To find the evidence \( Z \) What we're doing is essentially
evaluating a high-dimensional integral over the possible values of the
parameters (Let's assume that all the physical quantities are
normalised: \( \lbrace \theta \rbrace \in [0, 1] \)). Under such conditions, the integral
can be approximated by the Riemann Sum, with higher accuracy of the
integral as a result of a larger number of
points.

Needless to say that if we choose to rasterise with \( n \) points,
the number of samples for \(D \) dimensional parameter space is
\( n^{D} \). Thus these techniques are inefficient compared to Nested Sampling\cite{polychord}.

\subsection{Metropolis-Hastings and other forms of Markov-chain Monte-Carlo}\label{sec-3-2}

These approaches are a form of rejection sampling, thus only a subset
of points is used in the evaluation of the integral. Markov Chain
Monte-Carlo algorithms also suffer from arbitrary halting
criteria\cite{Metropolis,Metropolis-hastings-gibbs}. For
example, when simulating an Ising Spin array, the time taken to fully
equilibrate the system near critical temperature is not easily
predictable, and determining whether a system has equilibrated at
later stages is simply not practical.\cite{me-ising}

Another drawback to MC-MC methods is that they're not easily made
concurrent.\cite{Metropolis-hastings-gibbs} In my previous work, I have
shown that the multi-process scaling of such parallel implementations is
sub-optimal, and a more significant speedup can be obtained simply by
running several simulations as full POSIX processes in parallel.\cite{me-ising}

\subsection{The need for nested sampling}\label{sec-3-3}

As we can see, all naive approaches brute force the multidimensional
integral.  In other words, we may see a speedup by sampling from
representative points, and avoiding other non-representative samples
as much as possible.

This was the basis for the paper due to John
Skilling\cite{skilling2006}, on a new machine learning technique, that
allowed to minimise the number of so-called live points.

\section{Nested sampling}\label{sec-4}

There were multiple
improvements\cite{polychord,Feroz2009MultiNestAE,Higson2018DynamicNS}
on the original algorithm, so we shall only present the ideas and
avoid the unnecessary and obsolete details as much as possible.

We want to accumulate a quantity called \emph{prior mass} (for reasons that
will become clear soon), defined as:

\begin{equation}\label{eq:prior-mass}
X ( \lambda)  = \int_{L(\theta) > \lambda} \pi (\theta) d \theta,
\end{equation}
which is the cumulant prior mass covering all likelihood values
greater than \( \lambda \). Note that increasing \( \lambda \) decreases the value
for \( X( \lambda) \) from \(1\) to \(0\).

By exploiting the existence of the inverse function
\( L ( X ( \lambda) ) \equiv \lambda \), we can simplify the evidence to
\begin{equation}
Z = \int _0 ^1 L(X) dX.
\end{equation}
Thus by sampling randomly, given \(N\) points, we expect that the
approximate value of the integral~\eqref{eq:prior-mass} is given by a
Riemann sum:

\begin{equation}
  Z = \sum_i^{N^{2}} \frac{L_i}{N}.
\end{equation}\cite[See][]{skilling2006}~for a more detailed explanation and example.

Next we might be interested in the order of convergence of such a
method, so we are interested in the order of the error terms.

By using a more clever approximation for the
integral\cite{skilling2006} (e.g.~using a trapezoidal rule) we can
show that the error falls as \( O ( N^2 ) \).


\subsection{Basic algorithm}\label{sec-4-1}

We start by taking \( m \) random samples from the distribution
(i.e.~evaluating the likelihood for a single set of random physical
parameter values). We shall refer to them as \emph{live points}.

On each iteration, we take the point with the lowest likelihood and
record it. Then we pick new values for each of the parameters
according to some criteria, that have a higher likelihood than the
old point. The old point now becomes a \emph{dead point} while the new
value is added to the \emph{live points}.

As a result, a human observing the set of live points would notice
that the latter preferentially occupy areas of high
likelihood. Moreover there will be a contour of constant likelihood
passing through each of the points (dead and live) and the live points
will necessarily occupy the contours of highest likelihood.

Moreover, since we've picked points at random inside a \(D\)
dimensional hyper-cube, the volumes delineated by the contours will
correspond roughly to \( \frac{1}{m} \)-th of the total volume of the
hyper-cube. This allows us to estimate the probabilities delineated by
the contours and effectively evaluate the likelihood integral.

Moreover this being an approximate method, we can also estimate the
error within each probability and incorporate that uncertainty into
the analysis.

So bringing it all together we have the algorithm 1.
\begin{algorithm}
  \SetAlgoLined%
\textbf{Start} with  $\lbrace\theta_1, \ldots , \theta_N\rbrace \in \text{prior}$;

\textbf{initialise} $Z = 0, X_0 = 1$;

\For{$i = 1, 2, \ldots ,$ \KwTo $j$}{%
%
\textbf{set} $L_i = $ lowest of current likelihood values,

\textbf{set} $X_i = \exp(-i/N)$ (\emph{crude}) \tcc{or sample it to get uncertainty,}


\textbf{set} $w_i = X_{i-1} - X_i$ (\emph{simple}) \tcc{\textbf{or} $(X_{i-1} - X_{i+1} )/2$ (\emph{trapezoidal}),}


\textbf{increment} $Z$ \textbf{by} $L_i w_i$

\textbf{replace} point of lowest likelihood \textbf{with} new one drawn from within $L(\theta) > L_i$, \tcc{in proportion to the prior $\pi(\theta)$.}}

\textbf{increment} $Z$ \textbf{by} $N^{-1}$ ($L(\theta_1 ) + . . . + L(\theta_N )) X_j$.
\caption{Nested Sampling. Credit\cite{skilling2006}}
\end{algorithm}
\goodbreak%

\subsection{Improvements: Posterior re-partitioning}\label{sec-4-2}

Looking at the algorithm\cite{chen-ferroz-hobson} leads us to a
few important conclusions. The shape of the prior matters, and as such
if the prior and the posterior are the same, the algorithm should
converge more rapidly, than with a flat prior.Moreover the prior is a
function of what our parameters' definition. In other words a
co-ordinate transformation can change the shape of the prior. Thus a
clever choice of such a transformation of physical parameters
\( \theta \) can result in a significant speedup.

As to why it's called re-partitioning, because the true physical
quantities of interest are really only dependent on the product
\( L (\theta) \pi (\theta) \) and which is which (i.e.~where do we partition the
likelihood from the prior) is only a matter of choice, thus we can
easily come up with a more tractable form by moving the boundary ---
\( \tilde{\pi} (\theta) \) as long as the product
\( L (\theta ) \pi (\theta) = \tilde{L} (\theta) \tilde{\pi} (\theta) \) is the same.

For example\cite{chen-ferroz-hobson}, consider the following re-partitioning:

\begin{eqnarray} 
\tilde{\pi} (\theta) &=& \frac{\pi {(\theta)}^\beta }{Z_\pi (\beta)},\\
\tilde{L} (\theta) &=&L(\theta) \pi {(\theta)}^{(1-\beta)} Z_\pi(\beta) ,
\end{eqnarray}
where $Z_{\pi}(\beta) = \int \pi {(\theta)}^{\beta} d\theta$. 

In this particular case we expect a speedup due to the re-scaled prior
that is allowed to change according to the distribution, but with a
trade-off of adding another parameter to the fit. Of course, the
larger the number of already considered parameters, the more
favourable the trade-off.

Additionally this particular re-partitioning scheme doesn't produce
the expected result in the case where the prior has hard cutoffs.
\goodbreak%
\section{Goal of the project}\label{sec-5}

Thus the project shall need to be preoccupied with the following three
main avenues. First we shall aim to test Nested Sampling in
application to parameters which were generated using a toy
distribution and evaluating the effects of different choices of priors
and parameter values.

We shall then focus on attempting to improve the performance of said
nested sampling, by considering other methods of re-partitioning:
i.e.~attempting a similar trick on the posterior half of the Bayes'
theorem.

Accordingly we shall also investigate other re-partitioning
schemes. In particular we shall consider cases where there are more
parameters, and hard cutoffs, to cope with the limitations of the
partitioning presented in~\cite{chen-ferroz-hobson}.
\goodbreak%
More concretely I shall: 
\begin{itemize}
\item Run Polychord on data generated with an N-dimensional Correlated
  Gaussian and uniform prior.
\item Implement the same with a multivariate Gaussian prior, and check
  that it is indeed faster than with a uniform prior.
\item Investigate \emph{mixture-model} prior re-partitioning, i.e.~a
  re-partitioning scheme where
  \( \tilde{\pi} = \beta \pi + (1 - \beta ) \pi_\text{Gaussian}\)
  
\item Attempt to apply to real data from Planck\cite{planck}.
\end{itemize}

Achieving all of the above will make the project at least a partial
success. However, given enough time, and sufficient resources, there
could be extra work that would be useful for this project to be a
complete success.

\goodbreak%
\begin{itemize}
\item Investigate applicability of other mixture models,\cite{Higson2018DynamicNS}
\item Investigate applicability of re-partitioning schemes with a higher
  number of tuning parameters.
\item Investigate the potential improvements arising from performing
  said operations algorithmic at compile-time.
\end{itemize}
% \end{multicols}
\bibliography{bibliography}
%\printbibliography%
\end{document}
%  LocalWords:  dX eq
